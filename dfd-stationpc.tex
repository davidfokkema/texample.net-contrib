\documentclass{article}

\usepackage{tikz}
\usetikzlibrary{calc,arrows,decorations.pathmorphing,intersections}

\usepackage[font={small,sf},labelfont={bf},labelsep=endash]{caption}
\usepackage{sansmath}
\DeclareCaptionFormat{thesis}{\sansmath #1#2#3}
\captionsetup{format=thesis}
\usepackage{relsize}
\usepackage{xspace}

\usepackage{shape-datastore}

\newcommand{\hisparc}{\textsmaller{HiSPARC}\xspace}
\newcommand{\labview}{\textsmaller{LabVIEW}\xspace}

\begin{document}

\begin{figure}
\begin{tikzpicture}
[font=\sffamily,
 every matrix/.style={ampersand replacement=\&,column sep=2cm,row sep=2cm},
 source/.style={draw,thick,rounded corners,fill=yellow!20,inner sep=.3cm},
 process/.style={draw,thick,circle,fill=blue!20},
 sink/.style={source,fill=green!20},
 datastore/.style={draw,very thick,shape=datastore,inner sep=.3cm},
 dots/.style={gray,scale=2},
 to/.style={->,>=stealth',shorten >=1pt,semithick,font=\sffamily\footnotesize},
 every node/.style={align=center},
]
\matrix{
\node[source] (hisparcbox) {electronics}; \& \node[process] (daq) {DAQ};
\&
\\
\& \node[datastore] (buffer) {buffer}; \& \\
\node[datastore] (storage) {storage}; \& \node[process] (monitor) {monitor}; \& \node[sink] (datastore) {datastore}; \\
};

\draw[to] (hisparcbox) -- node[midway,above] {raw events} node[midway,below] {level 0} (daq);
\draw[to] (daq) -- node[midway,right] {raw event data\\level 1} (buffer);
\draw[to] (buffer) -- node[midway,right] {raw event data\\level 1} (monitor);
\draw[to] (monitor) to[bend right=50] node[midway,above] {events} node[midway,below] {level 1} (storage);
\draw[to] (storage) to[bend right=50] node[midway,above] {events} node[midway,below] {level 1} (monitor);
\draw[to] (monitor) -- node[midway,above] {events} node[midway,below] {level 1} (datastore);
\end{tikzpicture}
\caption{Data flow diagram of a \hisparc station PC.  A \labview
program communicates with the \hisparc II electronics.  Data is
sanitized and, along with preliminary analysis results and configuration
settings, sent to the \emph{buffer}.  The monitor program retrieves raw event
data from the buffer and creates structured array data.  The events are then stored in
the \emph{storage}.  When a batch of events is ready, it is retrieved from
the storage and uploaded over the internet to the datastore.}
\end{figure}

\end{document}